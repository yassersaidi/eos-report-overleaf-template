\chapter*{Résumé}
\lhead{}
\rhead{\textit{Résumé}}

La Segmentation d'images basée sur l'apprentissage en profondeur est désormais établie comme un outil robuste de segmentation d'images. Il a été largement utilisé pour séparer les zones homogènes en tant que premier élément critique du pipeline de diagnostic et de traitement.\newline

L'objectif de ce travail est de proposer une approche de Deep Learning dans le domaine de l'épidémiologie pour détecter les tumeurs cérébrales.\newline

Pour ce faire, nous avons choisi d'utiliser les réseaux de neurones convolutifs (CNN), où différents modèles ont été implémentés nous permettant d'obtenir les meilleurs résultats.\newline

Dans cette étude, nous avons proposé deux approches de réseau neuronal convolutif et créé deux modèles fondés sur ces approches, les modèles créés ont été initiés et ont prouvé leur efficacité en atteignant une précision élevée et un score F1 dans leurs tests à l'aide de données BraTS 2019.

\paragraph{Mots clés:}
Segmentation d'images, réseau de neurones convolutifs, traitement d'images, classification d'images, imagerie médicale, intelligence artificielle, apprentissage automatique, apprentissage en profondeur, réseaux de neurones.
\setstretch{1.3}

\newpage
\chapter*{Abstract}
\lhead{}
\rhead{\textit{Abstract}}

Deep learning-based image segmentation is by now firmly established as a robust tool in image segmentation. It has been widely used to separate homogeneous areas as the first and critical component of diagnosis and treatment pipeline.\newline

The objective of this work is to propose a Deep Learning approach in the field of epidemiology to detect brain tumer.\newline

To do this, we have chosen to use the Convolutional Neural networks (CNN), where different models have been implemented allowing us to obtain the best results.\newline

In this study, we proposed two Convolutional Neural Network approaches and crated two models based on these approaches, the created models were trained and prove their efficiency by achieving high accuracy and F1-Score in their testing using BraTS 2019 Data.
\paragraph{Key words:}
Image Segmentation,Convolutional neural network, image processing, image classification, Medical imaging ,Artificial intelligence, Machine Learning, Deep Learning, Neural networks.

\newpage

\chapter*{{\RL{الملخص}}}
\rhead{}
\rhead{\textit{الملخص}}

\vspace{10pt}
\begin{RLtext}
أصبحت عملية تجزئة الصور القائمة على التعلم العميق الآن كأداة قوية في تجزئة الصور. يتم استخدامها في نطاق واسع لفصل المناطق المتجانسة، و تعتبر مكون أول وحاسم في عمليات  التشخيص والعلاج.

\vspace{10pt}
الهدف من هذا العمل هو اقتراح نهج من التعلم العميق في مجال علم الأوبئة للكشف عن ورم الدماغ.

\vspace{10pt}
للقيام بذلك، اخترنا استخدام الشبكات العصبية التلافيفية\LR {CNN } ، حيث تم تنفيذ نماذج مختلفة مما يتيح لنا الحصول على أفضل النتائج. 

\vspace{10pt}
في هذه الدراسة، اقترحنا نهجين للشبكة العصبية التلافيفية وصممنا نموذجين بناءً على هذه الأساليب، وتم تدريب النماذج التي تم إنشاؤها وإثبات كفاءتها من خلال تحقيق دقة عالية و\LR{F1-Score} في اختبارها باستخدام بيانات\LR{BraTS 2019}.

\vspace{10pt}

\end{RLtext}

\begin{RLtext}
\textbf{الكلمات الدالة:}
التصوير
 الطبي، ورم الدماغ، الذكاء الاصطناعي، التعلم الآلي، التعلم العميق، الشبكات العصبية، الشبكة العصبية التلافيفية, معالجة الصور، تصنيف الصور،تجزئة الصور
 
 .
 
\end{RLtext}
