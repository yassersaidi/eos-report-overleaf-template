\chapter*{Introduction générale}
\addcontentsline{toc}{chapter}{Introduction générale}
\lhead{}
\rhead{\textit{Introduction générale}}

La Segmentation d’image consiste à séparer l’image en des zones simples avec une certaine homogénéité.  C’est une partition de l'image en ensembles de pixels homogènes selon un critère prédéfini (ou voxel dans les cas des images 3D). La segmentation n'est pas unique et ses résultats sont différents selon : les algorithmes utilisés, les critères d'homogénéité, l’initialisation.\newline


L'adaptabilité des systèmes de segmentation est le principal défi auquel sont confrontés les chercheurs dans le domaine de l'imagerie médicale.  La difficulté réside dans la complexité des images médicales. Ce sont souvent des images non photographiques comme c'est le cas pour les images IRM, qui sont générées à partir d'ondes électromagnétiques. Ces ondes sont émises par les noyaux d'hydrogènes lors de leurs relaxations. Ce type d'image est caractérisé par la présence de formes complexes dont l'apparence varie d'un sujet à un autre. On remarque, également, la présence d’un bruit assez considérable dans la plupart des cas, et une résolution pas toujours satisfaisante.\newline

Par ailleurs, le Deep Learning a récemment connu des succès opérationnels. Dans le traitement d’images médicales, notamment à travers les performances spectaculaires obtenues par les réseaux de neurones convolutifs (ConvNets) au challenge ImageNet 2012.


\newpage
 L’objectif de ce projet est de découvrir quelques approches de Deep Learning utilisées pour la segmentation d’images médicales, choisir une architecture de Deep Learning, réaliser son implémention et l'appliquer au dataset BraTS19.\newline

Notre mémoire est organisée comme suit :
\newline
\begin{itemize}


\item \textbf{Chapitre 1: Traitement d'image}\newline
Dans ce chapitre, nous avons traité les notions de base nécessaires et la compréhension des techniques de traitement d’images et quelques méthodes de segmentation.\newline

\item \textbf{Chapitre 2: Deep Learning}\newline
Dans ce chapitre,nous parlerons de certains approches De Deep Learning.\newline

\item \textbf{Chapitre 3: Les Architectures proposées pour la Segmentation d’images Biomédicales}\newline
Dans ce chapitre, nous parlerons de quelques algorithmes courants utilisés pour La segmentation d'images médicales.
\newline
\item  \textbf{Chapitre 4: Matériels et Méthodes}\newline
Ce chapitre sera divisé en deux sections, Dans la première section, nous parlerons de base de données que nous utiliserons, en mentionnant tous ses détails. Après cela, nous passons aux méthodes section, où nous parlerons de nos modèles proposés.
\newline 

\item  \textbf{Chapitre 5: Réalisation}\newline
Dans ce chapitre, nous allons implémenter nos modèles proposés et discuter des résultats
obtenu.

\end{itemize}
\vspace{1cm}
Enfin, nous terminons ce travail par une conclusion générale.

% La deuxième partie du projet consiste à appliquer un modèle Deep-Learning aux images médicales à travers l’implémentation d’un modèle (par exemple Unet) pour la segmentation de ces images. Le modèle sera entrainé et évalué sur la base BraTS19 Du CBICA (Center for Biomedical Image Computing & Analytics).


